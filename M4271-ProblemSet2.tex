\documentclass[11pt]{article}
\usepackage[utf8]{inputenc}
\usepackage[dvipsnames]{xcolor}
\usepackage[margin=1in]{geometry}
\usepackage{amsfonts, amssymb, amsmath, amsthm, booktabs, hyperref, pgfplots, tikz, xcolor, mathrsfs}

\theoremstyle{definition}\newtheorem{definition}{Definition}
\theoremstyle{definition}\newtheorem{question}{Question}
\theoremstyle{definition}\newtheorem*{solution}{Solution}
\theoremstyle{definition}\newtheorem{example}{Example}
\theoremstyle{definition}\newtheorem{notation}{Notation}
\theoremstyle{theorem}\newtheorem{theorem}{Theorem}
\theoremstyle{theorem}\newtheorem{corollary}{Corollary}
\theoremstyle{theorem}\newtheorem{lemma}{Lemma}
\theoremstyle{theorem}\newtheorem{proposition}{Proposition}

\newcommand{\A}{\mathcal{A}}
\newcommand{\B}{\mathcal{B}}
\newcommand{\C}{\mathbb{C}}
\newcommand{\CC}{\mathcal{C}}
\newcommand{\D}{\mathcal{D}}
\renewcommand{\d}{\delta}
\newcommand{\E}{\mathcal{E}}
\newcommand{\e}{\varepsilon}
\newcommand{\F}{\mathbb{F}}
\newcommand{\FF}{\mathcal{F}}
\newcommand{\G}{\mathcal{G}}
\renewcommand{\H}{\mathbb{H}}
\newcommand{\I}{\mathcal{I}}
\newcommand{\J}{\mathcal{J}}
\newcommand{\K}{\mathbb{K}}
\renewcommand{\L}{\mathscr{L}}
\newcommand{\M}{\mathcal{M}}
\newcommand{\N}{\mathbb{N}}
\renewcommand{\O}{\mathcal{O}}
\renewcommand{\P}{\mathcal{P}}
\newcommand{\Q}{\mathbb{Q}}
\newcommand{\R}{\mathbb{R}}
\renewcommand{\S}{\mathcal{S}}
\newcommand{\T}{\mathbb{T}}
\newcommand{\U}{\mathcal{U}}
\newcommand{\V}{\mathcal{V}}
\newcommand{\W}{\mathcal{W}}
\newcommand{\X}{\mathcal{X}}
\newcommand{\Y}{\mathcal{Y}}
\newcommand{\Z}{\mathbb{Z}}
\DeclareMathOperator{\Span}{span}

\everymath{\displaystyle}

\begin{document}

\noindent \textbf{MATH 4271 Dynamical Systems} \hfill \textbf{Problem Set 2} \\
\textsc{Practice} \hfill \textsc{Joe Tran}

\begin{question}
    \textcolor{red}{(Problem Set 1.2 Question 1, Perko)} Find the eigenvalues and eigenvectors of the matrix $A$ and show that $B = P^{-1}AP$ is a diagonal matrix. Solve the linear system $\frac{d\vec{y}}{dt} = B\vec{y}$ and then solve $\frac{d\vec{x}}{dt} = A\vec{x}$ using the corollary.
    \begin{itemize}
        \item[(a)] $A = \begin{bmatrix} 3 & 1 \\ 1 & 3 \end{bmatrix}$
        \item[(b)] $A = \begin{bmatrix} 1 & 3 \\ 3 & 1 \end{bmatrix}$
    \end{itemize}
\end{question}

\begin{solution}
    (a) We first find the eigenvalues of $A$ as follows:
    \begin{align*}
        \chi_A(\lambda) &= \det(A - \lambda I) \\
        &= \det\begin{bmatrix} 3 - \lambda & 1 \\ 1 & 3 - \lambda \end{bmatrix} \\
        &= (\lambda - 3)^2 - 1 \\
        &= (\lambda - 3 - 1)(\lambda - 3 + 1) \\
        &= (\lambda - 4)(\lambda - 2) = 0
    \end{align*}
    Therefore, our eigenvalues of $\lambda = 4$ and $\lambda = 2$. Now we find the eigenspaces. When $\lambda = 2$,
    \begin{align*}
        \E_A(2) &= \ker(A - 2I) \\
        &= \ker\left\{\begin{bmatrix} 1 & 1 \\ 1 & 1 \end{bmatrix}\right\} \\
        &= \ker\left\{\begin{bmatrix} 1 & 1 \\ 0 & 0 \end{bmatrix}\right\} \\
        &= \left\{\begin{bmatrix} -s \\ s \end{bmatrix} : s \in \R\right\} \\
        &= \Span\left\{\begin{bmatrix} -1 \\ 1 \end{bmatrix}\right\}
    \end{align*}
    Therefore, we may take $\vec{v}_1 = \begin{bmatrix} -1 \\ 1 \end{bmatrix}$ to be the eigenvector corresponding to $\lambda = 2$. When $\lambda = 4$,
    \begin{align*}
        \E_A(4) &= \ker(A - 4I) \\
        &= \ker\left\{\begin{bmatrix} -1 & 1 \\ 1 & -1 \end{bmatrix}\right\} \\
        &= \ker\left\{\begin{bmatrix} 1 & -1 \\ 0 & 0 \end{bmatrix}\right\} \\
        &= \left\{\begin{bmatrix} s \\ s \end{bmatrix} : s \in \R\right\} \\
        &= \Span\left\{\begin{bmatrix} 1 \\ 1 \end{bmatrix}\right\}
    \end{align*}
    Therefore, we may take $\vec{v}_2 = \begin{bmatrix} 1 \\ 1 \end{bmatrix}$ to be the eigenvector corresponding to $\lambda = 4$. With our eigenvectors, the matrix $P$ is given by
    \begin{equation*}
        P = \begin{bmatrix} \vec{v}_1 & \vec{v}_2 \end{bmatrix} = \begin{bmatrix} -1 & 1 \\ 1 & 1 \end{bmatrix}
    \end{equation*}
    and $P^{-1}$ is given by
    \begin{align*}
        P^{-1} &= \begin{bmatrix} -1 & 1 \\ 1 & 1 \end{bmatrix}^{-1} \\
        &= \frac{1}{\det(P)} \begin{bmatrix} 1 & -1 \\ -1 & -1 \end{bmatrix} \\
        &= -\frac{1}{2} \begin{bmatrix} 1 & -1 \\ -1 & -1 \end{bmatrix}
    \end{align*}
    and therefore, we can express $B = P^{-1}AP$ as
    \begin{equation*}
        \begin{bmatrix} 2 & 0 \\ 0 & 4 \end{bmatrix} = -\frac{1}{2}\begin{bmatrix} 1 & -1 \\ -1 & -1 \end{bmatrix} \begin{bmatrix} 3 & 1 \\ 1 & 3 \end{bmatrix} \begin{bmatrix} -1 & 1 \\ 1 & 1 \end{bmatrix}
    \end{equation*}
    Now we want to solve the equation $\frac{d\vec{y}}{dt} = \begin{bmatrix} 2 & 0 \\ 0 & 4 \end{bmatrix}\begin{bmatrix} y_1 \\ y_2 \end{bmatrix}$. Corresponding the coordinates, we have
    \begin{equation*}
        \frac{dy_1}{dt} = 2y_1 \quad \frac{dy_2}{dt} = 4y_2
    \end{equation*}
    So solving each equation individually, we have $y_1(t) = b_1e^{2t}$ and $y_2(t) = b_2e^{4t}$ and so the solution of the linear system is given by
    \begin{equation*}
        \vec{y}(t) = \begin{bmatrix} y_1(t) \\ y_2(t) \end{bmatrix} = \begin{bmatrix} b_1e^{2t} \\ b_2e^{4t} \end{bmatrix}
    \end{equation*}
    Finally, we solve the system given by $\frac{d\vec{x}}{dt} = A\vec{x}$. Using the Corollary, we have
    \begin{align*}
        \vec{x}(t) &= \begin{bmatrix} -1 & 1 \\ 1 & 1 \end{bmatrix}\begin{bmatrix} e^{2t} & 0 \\ 0 & e^{4t} \end{bmatrix} \begin{bmatrix} 1 & -1 \\ -1 & -1 \end{bmatrix}\begin{bmatrix} c_1 \\ c_2 \end{bmatrix} \\
        &= \begin{bmatrix} e^{2t} & -e^{4t} \\ -e^{2t} & -e^{-4t} \end{bmatrix} \begin{bmatrix} 1 & -1 \\ -1 & -1 \end{bmatrix} \begin{bmatrix} c_1 \\ c_2 \end{bmatrix} \\
        &= \begin{bmatrix} e^{2t} + e^{4t} & -e^{2t} - e^{4t} \\ -e^{2t} + e^{4t} & e^{2t} + e^{4t} \end{bmatrix} \begin{bmatrix} c_1 \\ c_2 \end{bmatrix}
    \end{align*}
    as required.

    (b) We first find the eigenvalues of $A$ as follows:
    \begin{align*}
        \chi_A(\lambda) &= \det(A - \lambda I) \\
        &= \det\begin{bmatrix} 1 - \lambda & 3 \\ 3 & 1 - \lambda \end{bmatrix} \\
        &= (\lambda - 1)^2 - 3^2 \\
        &= (\lambda - 1 - 3)(\lambda - 1 + 3) \\
        &= (\lambda - 4)(\lambda + 2) = 0
    \end{align*}
    Therefore, our eigenvalues are $\lambda = 4$ and $\lambda = -2$. Now we find the eigenspaces. When $\lambda = -2$,
    \begin{align*}
        \E_A(-2) &= \ker(A + 2I) \\
        &= \ker\left\{\begin{bmatrix} 3 & 3 \\ 3 & 3 \end{bmatrix}\right\} \\
        &= \ker\left\{\begin{bmatrix} 1 & 1 \\ 0 & 0 \end{bmatrix}\right\} \\
        &= \left\{\begin{bmatrix} -s \\ s \end{bmatrix} : s \in \R\right\} \\
        &= \Span\left\{\begin{bmatrix} -1 \\ 1 \end{bmatrix}\right\}
    \end{align*}
    So we may take $\vec{v}_1 = \begin{bmatrix} -1 \\ 1 \end{bmatrix}$ as one of our eigenvectors. When $\lambda = 4$,
    \begin{align*}
        \E_A(4) &= \ker(A - 4I) \\
        &= \ker\left\{\begin{bmatrix} -3 & 3 \\ 3 & -3 \end{bmatrix}\right\} \\
        &= \ker\left\{\begin{bmatrix} 1 & -1 \\ 0 & 0 \end{bmatrix}\right\} \\
        &= \left\{\begin{bmatrix} s \\ s \end{bmatrix} : s \in \R\right\} \\
        &= \Span\left\{\begin{bmatrix} 1 \\ 1 \end{bmatrix}\right\}
    \end{align*}
    So we may take $\vec{v}_2 = \begin{bmatrix} 1 \\ 1 \end{bmatrix}$ as one of our eigenvectors. With our two eigenvectors, the matrix $P$ is given by
    \begin{equation*}
        P = \begin{bmatrix} \vec{v}_1 & \vec{v}_2 \end{bmatrix} = \begin{bmatrix} -1 & 1 \\ 1 & 1 \end{bmatrix}
    \end{equation*}
    and $P^{-1}$ is given by
    \begin{equation*}
        P^{-1} = -\frac{1}{2}\begin{bmatrix} 1 & -1 \\ -1 & -1 \end{bmatrix}
    \end{equation*}
    and therefore, we can express $B = P^{-1}AP$ as
    \begin{equation*}
        \begin{bmatrix} -2 & 0 \\ 0 & 4 \end{bmatrix} = -\frac{1}{2}\begin{bmatrix} 1 & -1 \\ -1 & -1 \end{bmatrix} \begin{bmatrix} 1 & 3 \\ 3 & 1 \end{bmatrix} \begin{bmatrix} -1 & 1 \\ 1 & 1 \end{bmatrix}
    \end{equation*}
    Now we want to solve $\frac{d\vec{y}}{dt} = \begin{bmatrix} -2 & 0 \\ 0 & 4 \end{bmatrix} \begin{bmatrix} y_1 \\ y_2 \end{bmatrix}$. Corresponding the coordinates, we have
    \begin{equation*}
        \frac{dy_1}{dt} = -2y_1 \quad \frac{dy_2}{dt} = 4y_2
    \end{equation*}
    So solving each of equation individually, we have $y_1(t) = c_1e^{-2t}$ and $y_2(t) = c_2e^{4t}$ and so the solution of the linear system is given by
    \begin{equation*}
        \vec{y}(t) = \begin{bmatrix} c_1e^{-2t} \\ c_2e^{4t} \end{bmatrix}
    \end{equation*}
    Finally, we solve the system given by $\frac{d\vec{x}}{dt} = A\vec{x}$. Using the Corollary, we have
    \begin{align*}
        \vec{x}(t) &= \begin{bmatrix} 1 & -1 \\ -1 & -1 \end{bmatrix}\begin{bmatrix} e^{-2t} & 0 \\ 0 & e^{4t} \end{bmatrix} \begin{bmatrix} 1 & -1 \\ -1 & -1 \end{bmatrix} \begin{bmatrix} c_1 \\ c_2 \end{bmatrix} \\
        &= \begin{bmatrix} e^{-2t} & -e^{4t} \\ -e^{-2t} & -e^{4t} \end{bmatrix} \begin{bmatrix} 1 & -1 \\ -1 & -1 \end{bmatrix} \begin{bmatrix} c_1 \\ c_2 \end{bmatrix} \\
        &= \begin{bmatrix} e^{-2t} + e^{4t} & -e^{-2t} + e^{4t} \\ -e^{-2t} + e^{4t} & e^{-2t} + e^{4t} \end{bmatrix}\begin{bmatrix} c_1 \\ c_2 \end{bmatrix}
    \end{align*}
\end{solution}

\begin{question}
    \textcolor{red}{(Problem Set 1.2 Question 2, Perko)} Find the eigenvalues and the eigenvectors for the matrix $A$, solve the linear system $\frac{d\vec{x}}{dt} = A\vec{x}$, and determine the stable and unstable subspaces for the linear system.
    \begin{equation*}
        \frac{d\vec{x}}{dt} = \begin{bmatrix} 1 & 0 & 0 \\ 1 & 2 & 0 \\ 1 & 0 & -1 \end{bmatrix} \vec{x}
    \end{equation*}
\end{question}

\begin{solution}
    First finding the eigenvalues of $A$, we have
    \begin{align*}
        \chi_A(\lambda) &= \det(A - \lambda I) \\
        &= \begin{vmatrix} 1 - \lambda & 0 & 0 \\ 1 & 2 - \lambda & 0 \\ 1 & 0 & -1 - \lambda \end{vmatrix} \\
        &= (1 - \lambda)(2 - \lambda)(-1 - \lambda) = 0
    \end{align*}
    So we have eigenvalues $\lambda = 1$, $\lambda = 2$, and $\lambda = -1$. 
    
    Now we find the eigenspaces for each eigenvalue. For $\lambda = -1$,
    \begin{align*}
        \E_A(-1) &= \ker(A + I) \\
        &= \ker\left\{\begin{bmatrix} 2 & 0 & 0 \\ 1 & 3 & 0 \\ 1 & 0 & 0 \end{bmatrix}\right\} \\
        &= \ker\left\{\begin{bmatrix} 1 & 0 & 0 \\ 0 & 1 & 0 \\ 0 & 0 & 0 \end{bmatrix}\right\} \\
        &= \Span\left\{\begin{bmatrix} 0 \\ 0 \\ 1 \end{bmatrix}\right\}
    \end{align*}
    So we may take $\vec{v}_1 = \begin{bmatrix} 0 \\ 0 \\ 1 \end{bmatrix}$ as an eigenvector corresponding to $\lambda = -1$ for $A$. For $\lambda = 1$,
    \begin{align*}
        \E_A(1) &= \ker(A - I) \\
        &= \ker\left\{\begin{bmatrix} 0 & 0 & 0 \\ 1 & 1 & 0 \\ 1 & 0 & -2 \end{bmatrix}\right\} \\
        &= \ker\left\{\begin{bmatrix} 1 & 0 & -2 \\ 1 & 2 & 0 \\ 0 & 0 & 0 \end{bmatrix}\right\} \\
        &= \Span\left\{\begin{bmatrix} 2 \\ -2 \\ 1 \end{bmatrix}\right\}
    \end{align*}
    So we may take $\vec{v}_2 = \begin{bmatrix} 2 \\ -2 \\ 1 \end{bmatrix}$ as an eigenvector corresponding to $\lambda = 1$ for $A$. For $\lambda = 2$,
    \begin{align*}
        \E_A(2) &= \ker(A - 2I) \\
        &= \ker\left\{\begin{bmatrix} -1 & 0 & 0 \\ 1 & 0 & 0 \\ 1 & 0 & -3 \end{bmatrix}\right\} \\
        &= \ker\left\{\begin{bmatrix} 1 & 0 & 0 \\ 0 & 0 & 1 \\ 0 & 0 & 0 \end{bmatrix}\right\} \\
        &= \Span\left\{\begin{bmatrix} 0 \\ 1 \\ 0 \end{bmatrix}\right\}
    \end{align*}
    So we may take $\vec{v}_3 = \begin{bmatrix} 0 \\ 1 \\ 0 \end{bmatrix}$ as an eigenvector corresponding to $\lambda = 2$ for $A$.

    Now we find the matrices $P$ and $P^{-1}$. Indeed, with our eigenvectors
    \begin{equation*}
        P = \begin{bmatrix} 0 & 2 & 0 \\ 0 & -2 & 1 \\ 1 & 1 & 0 \end{bmatrix}
    \end{equation*}
    and thus,
    \begin{align*}
        P^{-1} &= -\frac{1}{2} \begin{bmatrix} 1 & 0 & -2 \\ -1 & 0 & 0 \\ -2 & - 2 & 0 \end{bmatrix}
    \end{align*}
    
    Therefore, the solution of the linear system is given as
    \begin{align*}
        \vec{x}(t) &= \begin{bmatrix} 0 & 2 & 0 \\ 0 & -2 & 1 \\ 1 & 1 & 0 \end{bmatrix}\begin{bmatrix} e^{-t} & 0 & 0 \\ 0 & e^t & 0 \\ 0 & 0 & e^{2t} \end{bmatrix} \begin{bmatrix} 1 & 0 & -2 \\ -1 & 0 & 0 \\ -2 & -2 & 0 \end{bmatrix}\begin{bmatrix} c_1 \\ c_2 \\ c_3 \end{bmatrix} \\
        &= \begin{bmatrix} 0 & 2e^t & 0 \\ 0 & -2e^t & e^{2t} \\ e^{-t} & e^t & 0 \end{bmatrix} \begin{bmatrix} 1 & 0 & -2 \\ -1 & 0 & 0 \\ -2 & -2 & 0 \end{bmatrix} \begin{bmatrix} c_1 \\ c_2 \\ c_3 \end{bmatrix} \\
        &= \begin{bmatrix} -2e^t & 0 & 0 \\ 2e^t - 2e^{2t} & -2e^{2t} & 0 \\ e^{-t} - e^t & 0 & -2e^{-t} \end{bmatrix}\begin{bmatrix} c_1 \\ c_2 \\ c_3 \end{bmatrix}
    \end{align*}
\end{solution}

\begin{question}
    \textcolor{red}{(Problem Set 1.2 Question 3, Perko)} Write the following linear differential equations with constant coefficients in the form of the linear system (1) and solve
    \begin{equation*}
        \frac{d^3x}{dt^3} - 2\frac{d^2x}{dt^2} - \frac{dx}{dt} + 2x = 0
    \end{equation*}
    (Hint: Let $x_1 = x$ and $x_2 = \frac{dx_1}{dt}$, etc.)
\end{question}

\begin{solution}
    Assume that $x(t) = e^{\alpha t}$ is a solution of the above equation. Then
    \begin{align*}
        e^{\alpha t}(\alpha^3 - 2\alpha^2 - \alpha + 2) &= 0 \\
        e^{\alpha t}[\alpha^2(\alpha - 2) - (\alpha - 2)] &= 0 \\
        e^{\alpha t}(\alpha^2 - 1)(\alpha - 2) &= 0 \\
        e^{\alpha t}(\alpha + 1)(\alpha - 1)(\alpha - 2) &= 0
    \end{align*}
    So we have $\alpha = -1$, $\alpha = 1$ and $\alpha = 2$. Therefore, the general solution of the above equation is
    \begin{equation*}
        x(t) = c_1e^{-t} + c_2e^t + c_3e^{2t}
    \end{equation*}
    as required.
\end{solution}

\begin{question}
    \textcolor{red}{(Problem Set 1.2 Question 4, Perko)} Using the corollary of this section, solve the initial value problem
    \begin{equation*}
        \frac{d\vec{x}}{dt} = A\vec{x} \quad \vec{x}(0) = \vec{x}_0
    \end{equation*}
    \begin{itemize}
        \item[(a)] With $A$ given by Question 1(a) and $\vec{x}_0 = (1, 2)^T$.
        \item[(b)] With $A$ given by Question 2 and $\vec{x}_0 = (1, 2, 3)^T$.
    \end{itemize}
\end{question}

\begin{solution}
    (a) We have the solution given as
    \begin{align*}
        \vec{x}(0) &= \begin{bmatrix} 1 \\ 2 \end{bmatrix} = \begin{bmatrix} 2 & -2 \\ 0 & 2 \end{bmatrix} \begin{bmatrix} c_1 \\ c_2 \end{bmatrix}
    \end{align*}
    Then solving the system, we obtain $\begin{bmatrix} c_1 \\ c_2 \end{bmatrix} = \begin{bmatrix} \frac{3}{2} \\ 1 \end{bmatrix}$ and so the unique solution of the system is given by
    \begin{equation*}
        \vec{x}(t) = \begin{bmatrix} e^{2t} + e^{4t} & -e^{2t} - e^{4t} \\ -e^{2t} + e^{4t} & e^{2t} + e^{4t} \end{bmatrix} \begin{bmatrix} \frac{3}{2} \\ 2 \end{bmatrix}
    \end{equation*}

    (b) We have the solution given as
    \begin{equation*}
        \vec{x}(0) = \begin{bmatrix} 1 \\ 2 \\ 3 \end{bmatrix} = \begin{bmatrix} -2 & 0 & 0 \\ 0 & -2 & 0 \\ 0 & 0 & -2 \end{bmatrix}\begin{bmatrix} c_1 \\ c_2 \\ c_3 \end{bmatrix}
    \end{equation*}
    Then solving the system, we obtain $\begin{bmatrix} c_1 \\ c_2 \\ c_3 \end{bmatrix} = \begin{bmatrix} -\frac{1}{2} \\ -1 \\ -\frac{3}{2} \end{bmatrix}$ and so the unique solution is given by
    \begin{equation*}
        \vec{x}(t) = \begin{bmatrix} -2e^t & 0 & 0 \\ 2e^t - 2e^{2t} & -2e^{2t} & 0 \\ e^{-t} - e^t & 0 & -2e^{-t} \end{bmatrix}\begin{bmatrix} -\frac{1}{2} \\ -1 \\ -\frac{3}{2} \end{bmatrix}
    \end{equation*}
\end{solution}

\begin{question}
    \textcolor{red}{(Problem Set 1.2 Question 5, Perko)} Let $A$ be the $n \times n$ matrix with real and distinct eigenvalues. Find conditions on the eigenvalues that are necessary and sufficient for $\lim_{t \to \infty} \vec{x}(t) = \vec{0}$, where $\vec{x}(t)$ is any solution of $\frac{d\vec{x}}{dt} = A\vec{x}$.
\end{question}

\begin{solution}
    As we have previously solved before, the general solution is given by
    \begin{equation*}
        \vec{x}(t) = P\begin{bmatrix} e^{\lambda_1t} & 0 & \cdots & 0 \\ 0 & e^{\lambda_2t} & \cdots & 0 \\ \vdots & \vdots & & \vdots \\ 0 & 0 & \cdots & e^{\lambda_n t} \end{bmatrix}P^{-1}\vec{c}
    \end{equation*}
    Therefore, in order for $\lim_{t \to \infty} \vec{x}(t) = \vec{0}$, we need that all eigenvalues $\lambda_i < 0$.
\end{solution}

\end{document}
