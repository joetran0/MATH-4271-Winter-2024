\documentclass[11pt]{article}
\usepackage[utf8]{inputenc}
\usepackage{amsfonts, amssymb, amsmath, amsthm, booktabs, hyperref, pgfplots, tikz, xcolor, mathrsfs}

\theoremstyle{definition}\newtheorem{definition}{Definition}
\theoremstyle{definition}\newtheorem{notation}{Notation}
\theoremstyle{definition}\newtheorem{example}{Example}
\theoremstyle{theorem}\newtheorem{theorem}{Theorem}
\theoremstyle{theorem}\newtheorem{corollary}{Corollary}
\theoremstyle{theorem}\newtheorem{proposition}{Proposition}
\theoremstyle{theorem}\newtheorem{lemma}{Lemma}
\theoremstyle{theorem}\newtheorem{question}{Question}
\theoremstyle{remark}\newtheorem{remark}{Remark}

\newcommand{\K}{\mathbb{K}}
\newcommand{\C}{\mathbb{C}}
\newcommand{\CC}{\mathcal{C}}
\newcommand{\R}{\mathbb{R}}
\newcommand{\E}{\mathcal{E}}
\newcommand{\Q}{\mathbb{Q}}
\newcommand{\Z}{\mathbb{Z}}
\newcommand{\N}{\mathbb{N}}
\newcommand{\F}{\mathbb{F}}
\renewcommand{\SS}{\mathcal{S}}
\newcommand{\T}{\mathcal{T}}
\newcommand{\I}{\mathcal{I}}
\newcommand{\M}{\mathcal{M}}
\renewcommand{\L}{\mathscr{L}}
\newcommand{\e}{\varepsilon}
\newcommand{\teq}{\trianglelefteq}
\DeclareMathOperator{\Char}{char}
\DeclareMathOperator{\diag}{diag}
\DeclareMathOperator{\Span}{span}

\everymath{\displaystyle}
\setlength{\parskip}{5pt}

\begin{document}

\noindent \textbf{MATH 4271 Dynamical Systems} \hfill \textbf{Lecture 5} \\
\textsc{Lecture} \hfill \textsc{Joe Tran}

\section{Exponentials of Linear Operators and Matrices}

The ultimate objective in this section is to be able to write the general solution of $\frac{d\vec{x}}{dt} = A\vec{x}$, where $\vec{x}(0) = \vec{x}_0$ as
\begin{equation*}
    \vec{x}(t) = \vec{x}_0e^{At}
\end{equation*}
How do we compute $e^{At}$ or for any linear operator $T$, $e^T$?

Recall that an operator $T$ on a vector space $V$ is a mapping that maps $\vec{v} \in V$ to $\vec{w} = T(\vec{v}) \in V$, i.e. $T : V \to V$, or $\vec{v} \mapsto T(\vec{v})$.

\begin{definition}
    We call $L$ to be a linear operator if
    \begin{itemize}
        \item For any $\vec{v}, \vec{w} \in V$, $T(\vec{v} + \vec{w}) = T(\vec{v}) + T(\vec{w})$.
        \item For any $\alpha \in \K$ and $\vec{v} \in V$, then $T(\alpha \vec{v}) = \alpha \in T(\vec{v})$.
    \end{itemize}
\end{definition}

If $V = \R^n$, then $T : \R^n \to \R^n$, or $\vec{x} \mapsto T(\vec{x})$.

\begin{definition}
    The operator norm of $T$ is defined by
    \begin{equation*}
        \|T\|_{\infty} = \sup_{\|\vec{x}\|_2 \leq 1} |T(\vec{x})|
    \end{equation*}
    where
    \begin{equation*}
        \|\vec{x}\|_2 = \sqrt{x_1^2 + x_2^2 + \cdots + x_n^2}
    \end{equation*}
    is the Euclidean norm of $\vec{x} \in \K^n$.
\end{definition}

\begin{example}
    Let $T(\vec{x}) = A\vec{x}$, and let $A = \begin{bmatrix} 2 & 1 \\ 0 & 1 \end{bmatrix}$ and $\vec{x} = \begin{bmatrix} x_1 \\ x_2 \end{bmatrix}$. The $T(\vec{x})$ is given by
    \begin{equation*}
        T(\vec{x}) = A\vec{x} = \begin{bmatrix} 2 & 1 \\ 0 & 1 \end{bmatrix} \begin{bmatrix} x_1 \\ x_2 \end{bmatrix} = \begin{bmatrix} 2x_1 + x_2 \\ x_2 \end{bmatrix} \in \R^2
    \end{equation*}
    Now how do we find $\|T\|_{\infty}$, i.e.
    \begin{equation*}
        \|T\|_{\infty} = \sup_{\|\vec{x}\|_2 \leq 1} |T(\vec{x})| = \sup_{\|\vec{x}\|_2 \leq 1} |A\vec{x}| = \sup_{\|\vec{x}\|_2 \leq 1} \sqrt{(2x_1 + x_2)^2 + x_2^2}
    \end{equation*}
\end{example}

We use the method of Lagrange's multiplier to find the max of a function of two variables. Let
\begin{equation*}
    f(x_1, x_2) = \|T\|^2 = (2x_1 + x_2)^2 + x_2^2 = 4x_1^2 + 4x_1x_2 + x_2^2 + x_2^2
\end{equation*}

\begin{definition}
    A sequence of operators $(T_k)_{k = 1}^{n}$ is said to converge to an operator $T \in L(\K^n)$ if for every $\e > 0$, there exists an $N \in \N$ such that $\|T_k - T\| < \e$ for all $k \geq N$.
\end{definition}

\begin{notation}
    We use the notation $\L(\K^n)$ to be the linear space of linear operators on $\K^n$.
\end{notation}

\begin{proposition}
    Let $T, S \in \L(\K^n)$ and $A, B \in \M_n(\K)$. Then
    \begin{enumerate}
        \item $\|TS\| \leq \|T\| \|S\|$ or $\|AB\| \leq \|A\| \|B\|$
        \item $\|T^k\| \leq \|T\|^k$ for all $k = 0, 1, 2,...$. or $\|A^k\| \leq \|A\|^k$ for all $k = 0, 1, 2,...$.
    \end{enumerate} 
\end{proposition}

\begin{theorem}
    Let $T \in \L(\K^n)$ and let $t_0 > 0$. Then the series
    \begin{equation*}
        \sum_{n = 0}^{\infty} \frac{T^n t^n}{n!}
    \end{equation*}
    is absolutely and uniformly convergent.
\end{theorem}

Recall that $\sum_{n = 1}^{\infty} x_n$ is a convergent series if the sequence of partial sums $(s_n)_{n \in \N}$ is convergent, where for each $n \in \N$, $s_n = \sum_{k = 1}^{n} x_k$.

Recall that $\sum_{n = 1}^{\infty} x_n$ is said to be absolute convergent if $\sum_{n = 1}^{\infty} |x_n|$ converges.

\begin{definition}
    The exponential of $T \in \L(\K^n)$ is defined as
    \begin{equation*}
        e^T = \sum_{n = 0}^{\infty} \frac{T^n}{n!}
    \end{equation*}
\end{definition}

In particular, if $T : \K^n \to \K^n$, or $A = [a_{ij}]_{n \times n}$, then
\begin{equation*}
    e^{At} = \sum_{k = 0}^{n} \frac{A^kt^k}{k!}
\end{equation*}

Recall from calculus, the Taylor series tells us for $\theta \in \R^n$,
\begin{align*}
    e^\theta = \sum_{k = 0}^{n} \frac{\theta^k}{k!}
\end{align*}

\begin{example}
    Let $D = \begin{bmatrix} a & 0 \\ 0 & a \end{bmatrix}$. Find $e^D$.
\end{example}

To find $e^D$, observe that
\begin{align*}
    e^D &= \sum_{n = 0}^{\infty} \frac{D^n}{n!} \\
    &= I + \begin{bmatrix} a & 0 \\ 0 & a \end{bmatrix} + \frac{1}{2!}\begin{bmatrix} a^2 & 0 \\ 0 & a^2 \end{bmatrix} + \cdots \\
    &= \begin{bmatrix}
        1 + a + \frac{1}{2!}a^2 + \cdots & 0 \\
        0 & 1 + a + \frac{1}{2!}a^2 + \cdots
    \end{bmatrix} \\
    &= \begin{bmatrix} \sum_{n = 0}^{\infty} \frac{a^n}{n!} & 0 \\ 0 & \sum_{n = 0}^{\infty} \frac{a^n}{n!} \end{bmatrix} = \sum_{n = 0}^{\infty} \frac{a^n}{n!} I = e^aI
\end{align*}

\end{document}