\documentclass[11pt]{amsart}
\usepackage[utf8]{inputenc}
\usepackage[margin=1in]{geometry}
\usepackage{amsfonts, amssymb, amsmath, amsthm, booktabs, hyperref, pgfplots, tikz, xcolor}

\theoremstyle{definition}\newtheorem{definition}{Definition}
\theoremstyle{definition}\newtheorem{notation}{Notation}
\theoremstyle{definition}\newtheorem{example}{Example}
\theoremstyle{theorem}\newtheorem{theorem}{Theorem}
\theoremstyle{theorem}\newtheorem{corollary}{Corollary}
\theoremstyle{theorem}\newtheorem{proposition}{Proposition}
\theoremstyle{theorem}\newtheorem{lemma}{Lemma}
\theoremstyle{theorem}\newtheorem{question}{Question}
\theoremstyle{remark}\newtheorem{remark}{Remark}

\newcommand{\K}{\mathbb{K}}
\newcommand{\C}{\mathbb{C}}
\newcommand{\CC}{\mathcal{C}}
\newcommand{\R}{\mathbb{R}}
\newcommand{\Q}{\mathbb{Q}}
\newcommand{\Z}{\mathbb{Z}}
\newcommand{\N}{\mathbb{N}}
\newcommand{\F}{\mathbb{F}}
\renewcommand{\SS}{\mathcal{S}}
\newcommand{\T}{\mathcal{T}}
\newcommand{\I}{\mathcal{I}}
\newcommand{\M}{\mathcal{M}}
\newcommand{\teq}{\trianglelefteq}

\title{MATH 3022 Algebra II: Lecture 1}
\author{Joe Tran}
\date{Winter 2024}

\everymath{\displaystyle}

\setlength{\parindent}{0pt}
\setlength{\parskip}{5pt}

\begin{document}

\textbf{MATH 4271 Dynamical Systems} \hfill \textbf{Lecture 1} \\
\textsc{Lecture} \hfill \textsc{Joe Tran}

\section{Course Information}

\textbf{Course Weighting (TBD):}
\begin{itemize}
    \item Assignment 1: Homework Assignment via Crowdmark (20\%)
    \item Assignment 2: Reading Assignment and Group Presentation (25\%)
    \item Midterm Test (25\%)
    \item Final Exam (30\%)
\end{itemize}

\textbf{Main topics in this class:}
\begin{enumerate}
    \item Solving linear systems
    \item What is a dynamical system: general introduction, basic concepts and definitions, classification of dynamical systems
    \begin{itemize}
        \item Dynamical flows
        \item Qualitative analysis of scalar autonomous dynamical system (1D Flows)
        \item Some definitions
    \end{itemize}
    \item Phase Plane Analysis of Linear Systems
    \begin{itemize}
        \item Classification of equilibrium
        \item Local stability analysis and phase plane
        \item Extension to 3D linear systems
    \end{itemize}
    \item Dynamical theory of linear and nonlinear autonomous ODEs: existence, uniqueness
    \item Nonlinear systems and local and global stability.
    \item Theory of Bifurcation
    \item Oscillations in nonlinear systems
    \item Applications in ecology and epidemiology
    \item Discrete systems and chaos
\end{enumerate}

\textbf{Textbook:} Perko, Lawrence, \emph{Differential equations and dynamical systems}. Vol. 7, Springer Science \& Business Media, 2013.

\section{Linear Systems of ODEs}

The goal in this section is to be able to solve systems of linear ODEs. Furthermore, the goal is the study the qualitative behaviour of linear systems of ODEs.

\begin{definition}
    An ordinary differential equation (or system of ODEs) is said to be an \emph{autonomous ODE} if it does not explicitly depend on the independent variable. In other words, if the equation is written given in the form $\frac{dy}{dx} = (f \circ y)(x)$, that is, $x$ does not appear explicitly on the right side of the equation.
\end{definition}

\begin{example}
    Consider the one-dimensional autonomous ordinary differential equation given by
    \begin{equation*}
        \frac{dy}{dx} = f(x)
    \end{equation*}
    But if $x = x(t)$, then we may write
    \begin{equation*}
        \frac{dy}{dx} = f(x(t))
    \end{equation*}
    For example, we can write $\frac{dy}{dx} = 2y$, where $x$ is the independent variable, or we can also write $\frac{dx}{dt} = -3x^2 + 6$, where $t$ is the independent variable, in which both do not appear. Both of these examples are autonomous.

    As another example, consider $\frac{dx}{dt} = -3x^2 + 6t$. Then this is not an autonomous ODE because the function depends on the independent variable.
\end{example}

\begin{definition}
    A non-homogeneous $n \times n$ system of linear ODEs can be written as follows
    \begin{align*}
        \frac{dx_1}{dt} &= a_{11}x_1 + a_{12}x_2 + \cdots + a_{1n}x_n + b_1 \\
        \frac{dx_2}{dt} &= a_{21}x_1 + a_{22}x_2 + \cdots + a_{2n}x_n + b_2 \\
        &\vdots \\
        \frac{dx_n}{dt} &= a_{n1}x_1 + a_{n2}x_2 + \cdots + a_{nn}x_n + b_n \\
    \end{align*}
    or as a vector,
    \begin{equation*}
        \frac{d\vec{x}}{dt} = A\vec{x} + \vec{b}
    \end{equation*}
    where $\vec{x} = (x_1, x_2,..., x_n)^T$, $\vec{b} = (b_1, b_2,..., b_n)^T$, and $A = \begin{bmatrix} a_{11} & a_{12} & \cdots & a_{1n} \\ a_{21} & a_{22} & \cdots & a_{2n} \\ \vdots & \vdots & & \vdots \\ a_{n1} & a_{n2} & \cdots & a_{nn} \end{bmatrix}$, and $t \geq 0$.
\end{definition}

\begin{definition}
    A homogeneous $n \times n$ system of linear ODEs can be written as
    \begin{align*}
        \frac{dx_1}{dt} &= a_{11}x_1 + a_{12}x_2 + \cdots + a_{1n}x_n \\
        \frac{dx_2}{dt} &= a_{21}x_1 + a_{22}x_2 + \cdots + a_{2n}x_n \\
        &\vdots \\
        \frac{dx_n}{dt} &= a_{n1}x_1 + a_{n2}x_2 + \cdots + a_{nn}x_n \\
    \end{align*}
    or as a vector,
    \begin{equation*}
        \frac{d\vec{x}}{dt} = A\vec{x}
    \end{equation*}
    where $\vec{x} = (x_1, x_2,..., x_n)^T$, and $A = \begin{bmatrix} a_{11} & a_{12} & \cdots & a_{1n} \\ a_{21} & a_{22} & \cdots & a_{2n} \\ \vdots & \vdots & & \vdots \\ a_{n1} & a_{n2} & \cdots & a_{nn} \end{bmatrix}$, and $t \geq 0$.\footnote{Homogeneous system only occurs whenever $\vec{b} = \vec{0}$.}
\end{definition}

\begin{notation}
    We may express the system as follows:
    \begin{equation*}
        \frac{d\vec{x}}{dt} = A\vec{x} \quad \vec{x}' = A\vec{x}
    \end{equation*}
\end{notation}

\begin{example}
    Consider the system given as follows:
    \begin{align*}
        \frac{dx_1}{dt} &= -2x_2 \\
        \frac{dx_2}{dt} &= -3x_1
    \end{align*}
    We can write the system as
    \begin{equation*}
        \frac{d\vec{x}}{dt} = \begin{bmatrix} \frac{dx_1}{dt} \\ \frac{dx_2}{dt} \end{bmatrix} = \begin{bmatrix} 0 & -2 \\ -3 & 0 \end{bmatrix} \begin{bmatrix} x_1 \\ x_2 \end{bmatrix} = A\vec{x}
    \end{equation*}
\end{example}

\begin{example}
    Consider the system given by
    \begin{align*}
        \frac{dx_1}{dt} = -2x_1 \\
        \frac{dx_2}{dt} = -3x_2
    \end{align*}
    Then we can write the system as
    \begin{equation*}
        \frac{d\vec{x}}{dt} = \begin{bmatrix} \frac{dx_1}{dt} \\ \frac{dx_2}{dt} \end{bmatrix} = \begin{bmatrix} -2 & 0 \\ 0 & -3 \end{bmatrix} \begin{bmatrix} x_1 \\ x_2 \end{bmatrix} = A\vec{x}
    \end{equation*}
    Such systems of this form are called \emph{uncoupled systems}. It is elementary to find the general solutions of uncoupled systems. To solve this system, observe that
    \begin{align*}
        \frac{dx_1}{dt} &= -2x_1 \\
        \frac{1}{x_1} dx_1 &= -2 dt \\
        \int \frac{1}{x_1} dx_1 &= -2 \int dt \\
        \ln|x_1| &= -2t + c \\
        x_1 &= c_1e^{-2t}
    \end{align*}
    and similarly, $x_2 = c_2e^{-3t}$, so the general solution of $\vec{x}$ is
    \begin{equation*}
        \begin{bmatrix} x_1 \\ x_2 \end{bmatrix} = \begin{bmatrix} c_1e^{-2t} \\ c_2e^{-3t} \end{bmatrix}
    \end{equation*}
\end{example}

\begin{remark}
    If $\vec{x}_1(t)$ and $\vec{x}_2(t)$ are two solutions of the homogeneous equation $\frac{d\vec{x}}{dt} = A\vec{x}$, then any linear combination
    \begin{equation*}
        \vec{x}(t) = c_1\vec{x}_1(t) + c_2\vec{x}_2(t)
    \end{equation*}
    is also a solution (by the superposition principle).
\end{remark}

\begin{remark}
    If $\vec{x}_1(t)$ and $\vec{x}_2(t)$ are linearly independent vector solutions of the ordinary differential equation, then the general solution is written as
    \begin{equation*}
        \vec{x}(t) = c_1\vec{x}_1(t) + c_2\vec{x}_2(t)
    \end{equation*}
\end{remark}

In order to determine whether the vectors $\vec{x}_1(t)$ and $\vec{x}_2(t)$ are linearly independent, then we would have to either
\begin{itemize}
    \item Find the eigenvalue solutions of the vectors
    \item Show that the determinant is nonzero.
\end{itemize}

\section{Eigenvalue and Eigenvector Method}

Consider the scalar ordinary differential equation of the form
\begin{equation*}
    \frac{dx}{dt} = ax
\end{equation*}
where $a \neq 0$. Then the general solution is $x(t) = ce^{at}$, where $c \neq 0$ (to avoid triviality).

Now consider, in general, 
\begin{equation}
    \frac{d\vec{x}}{dt} = A\vec{x}
\end{equation}
Since (1) is linear and homogeneous, we can assume that $\vec{x}(t) = \vec{c}e^{\lambda t}$ is a solution of (1), where $\lambda \in \K$.\footnote{$\K$ is the notation to describe either $\R$ or $\C$} We will write now
\begin{equation}
    \vec{x}(t) = \vec{v}e^{\lambda t}
\end{equation}
Then
\begin{equation*}
    \frac{d\vec{x}}{dt} = \lambda \vec{v} e^{\lambda t}
\end{equation*}
So putting (2) into (1),
\begin{equation*}
    \lambda \vec{v}e^{\lambda t} = A\vec{v}e^{\lambda t}
\end{equation*}
and therefore,
\begin{equation*}
    (A - \lambda I_n)\vec{v} = \vec{0}
\end{equation*}
In this case, either $\vec{v} = \vec{0}$, or $A - \lambda I_n$. If $\vec{v} = \vec{0}$, we obtain a trivial solution. Otherwise, $A - \lambda I_n = \vec{0}$ and so $\det(A - \lambda I_n) = 0$.

\begin{center}
    -------------------------------------------------------- $\bullet$ --------------------------------------------------------
\end{center}


\end{document}